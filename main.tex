\documentclass[
    left=2.5cm,         % Sadly, generic margin parameter
    right=2.5cm,        % doesnt't work, as it is
    top=2.5cm,          % superseded by more specific
    bottom=3cm,         % left...bottom parameters.
    bindingoffset=6mm,  % Optional binding offset.
    nohyphenation=false % You may turn off hyphenation, if don't like. 
]{eiti/eiti-thesis}

\usepackage[polish]{babel}
\usepackage[
    backend=bibtex,
    style=ieee
]{biblatex}
\usepackage{csquotes}

\graphicspath{{img/}}             % Katalog z obrazkami.
\addbibresource{bibliografia.bib} % Plik .bib z bibliografią

%----------------------------------------
% Twierdzenia i definicje;
% tutaj ew. tłumaczymy te terminy
% na inne języki
%----------------------------------------
\newtheorem{theorem}{Twierdzenie}
\newtheorem{lemma}{Lemat}
\newtheorem{corollary}{Wniosek}
\newtheorem{definition}{Definicja}
\newtheorem{axiom}{Aksjomat}
\newtheorem{assumption}{Założenie}

%----------------------------------------
% Spis rysunków, tablic i załączników;
% tutaj ew. tłumaczymy te terminy
% na inne języki
%----------------------------------------
\AtBeginDocument{
    \renewcommand{\listfigurename}{Spis rysunków}
    \renewcommand{\listtablename}{Spis tabel}
    \renewcommand{\tablename}{Tabela}
}

\begin{document}

%--------------------------------------
% Strona tytułowa
%--------------------------------------
\EngineerThesis % dla pracy inżynierskiej mamy \EngineerThesis
\instytut{Automatyki i Informatyki Stosowanej}
\kierunek{Automatyka i Robotyka}

\title{
     System rozproszonej i bezpiecznej identyfikacji użytkownika urządzenia Internetu Rzeczy
}
\engtitle{ % Tytuł po angielsku do angielskiego streszczenia
    System for distributed and secure identification for the user of IoT device
}

\author{Dawid Kiciński}
\album{277115}
\promotor{mgr Maciej Stefańczyk}
\date{\the\year}
\maketitle

%--------------------------------------
% Streszczenie po polsku
%--------------------------------------
\streszczenie \lipsum[1-3]
\slowakluczowe XXX, XXX, XXX
\newpage

%--------------------------------------
% Streszczenie po angielsku
%--------------------------------------
\abstract \kant[1-3]
\keywords XXX, XXX, XXX
\newpage

%--------------------------------------
% Oświadczenie o autorstwie
%--------------------------------------
\makeauthorship
\blankpage

%--------------------------------------
% Spis treści
%--------------------------------------
\thispagestyle{empty}
\tableofcontents
\blankpage

%--------------------------------------
% Rozdziały
%--------------------------------------
\newpage
\section[Wstęp]{Wstęp}
To jest jakiś tekst po polsku. 

\section{Rozproszone uczenie na urządzeniu końcowym}

\subsection{Algorytm aktualizacji modelu globalnego}


\section{System weryfikacji użytkownika}

Zadaniem systemu jest weryfikacja użytkownika, cz zostały przedstiawione w następnej sekcji.

\subsection{Wstępne przetwarzanie obrazu}

\subsection{Weryfikacja użytkownika}
Weryfikacja twarzy jest zadaniem przyrównania twarzy kandydata to innej, i weryfikacja czy nastąpiło ich dopasowanie. Jest to mapowanie jeden-do-jednego: należy sprawdzić czy jest to ta sama osoba.

\subsubsection{Procedura weryfikacji}
\subsubsection{Ekstraktor cech}


% \input{tex/1-wstep}         % W długich pracach
% \input{tex/2-de-finibus}    % wygodnie jest trzymać
% \input{tex/3-code-listings} % każdy rozdział w osobnym pliku. 

%--------------------------------------------
% Literatura
%--------------------------------------------
\newpage
\printbibliography

%--------------------------------------------
% Spisy (opcjonalne)
%--------------------------------------------
\newpage

% Wykaz symboli i skrótów.
% Pamiętaj, żeby posortować symbole alfabetycznie
% we własnym zakresie. Ponieważ mało kto używa takiego wykazu, 
% uznałem, że robienie automatycznie sortowanej listy
% na poziomie LaTeXa to za duży overkill. 
% Jest tylko proste i oczywiste makro \acronym, 
% które dodaje postawowe formatowanie. 
% //AB
\vspace{0.8cm}
\section*{Wykaz symboli i skrótów}
\acronym{EiTI}{Wydział Elektroniki i Technik Informacyjnych}
\acronym{PW}{Politechnika Warszawska}

\listoffigures              % Spis obrazków. 
\vspace{1cm}                % vertical space
\listoftables               % Spis tabel. 
\vspace{1cm}                % vertical space
\listofappendices           % Spis załączników

% Załączniki
\appendix

\newpage
\newappendix{Nazwa załącznika 1}
\lipsum[1]

\newpage
\newappendix{Nazwa załącznika 2}
\lipsum[1]

% Używając powyższych spisów jako szablonu,
% możesz tu dodać swój własny wykaz bądź listę, 
% np. spis algorytmów. 

\end{document} % Dobranoc. 

