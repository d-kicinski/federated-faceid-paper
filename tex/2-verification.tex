\section[verification]{Systemy weryfikacji użytkownika za pomocą biometrii twarzy}
Zadaniem systemu jest weryfikacja użytkownika, cz zostały przedstiawione w następnej sekcji.

\subsection{Wstępne przetwarzanie obrazu}

\subsection{Weryfikacja użytkownika}

Weryfikacja twarzy jest zadaniem przyrównania twarzy
kandydata to innej, i weryfikacja czy nastąpiło ich dopasowanie. Jest to mapowanie
jeden-do-jednego: należy sprawdzić czy jest to ta sama osoba.

\subsubsection{Procedura weryfikacji} 

\subsubsection{Ekstraktor cech}
tutaj opisz jaki jest pipeline  
wchodzi zdjęcie i wychodzi embedding

\subsubsection{Zbiory danych}
Są jakieś zbiory daych 
1) VggFace2
2) MS1M
3) MegaFace
4) LWF

Sprawdzić wpływ datasetu początkowego na wyniki ewaluacji początkowego modelu globalego

ms1m do pre trengowania ponieważ posiada duża różnorodność twarzy oraz ogólną duża liczbe przykładów trenujących. VggFace2 zostonie z koleji wykorzystany do koncowego dotreniowania modelu - ze względu na dużą liczę zdjec przypadajaca na jedna osobę posiada dużą różnorodność wewnątrzklasową co powinno zwięszyć finałową wfektywaność podelu - większa odporność na rotację twarzy względem obiektywu, zmienne wartunki świetlne itp.

Ze względu na popularność dwóch ostatnich zbiorów w ewaluacji systemów rozpoznających twarze zostaną one właśnie wykorzystane do ewaluacji, MegaFace ma tę dodatkową zaletę, dużej liczby osób, a dwa pierwsze tylko do treningu.



\subsubsection{Metody weryikacji twarzy}
Wyjściem z modelu jest wektor(embedding), który pozwala na odróżnienie jednej twarzy od drugiej za pomocą porównanie 
embedingPapier googla twierdzi ze udało im sie uzyskać dobry score jednak nie rzadne inne prace nie raportują sukcesywngeo otrzymania podobnego wyniku. Może to wynikać z 1) braku zbliżonego wielkością zbioru danych użytych do uzyskania wyniku raportowangeo przez googla  2) brakiem podobnych zasobów obliczeniowychu jednej twarzy do embedingu drugiej twarzy.
Coś o tym, że są tak jakby dwie rodziny algorytmów. Jedna bazująca na klasycznym podjeściu stosowanym podczas klasyfikacji obrazów i druga bazująca na optymalizacji multi-class classiication hindge loss(jak to do uja wafla przetłumaczyć?).

Papier googla twierdzi ze udało im sie uzyskać dobry score jePapier googla twierdzi ze udało im sie uzyskać dobry score jednak nie rzadne inne prace nie raportują sukcesywngeo otrzymania podobnego wyniku. Może to wynikać z 1) braku zbliżonego wielkością zbioru danych użytych do uzyskania wyniku raportowangeo przez googla  2) brakiem podobnych zasobów obliczeniowychdnak nie rzadne inne prace nie raportują sukcesywngeo otrzymania podobnego wyniku. Może to wynikać z 1) braku zbliżonego wielkością zbioru danych użytych do uzyskania wyniku raportowangeo przez googla  2) brakiem podobnych zasobów obliczeniowych

\paragraph{Tripletloss}
jest o wiele lepszy dla zastosowań federated learningu
procedura trenowania


\paragraph{Classification loss}
Pokazać że jest kilka rodzajów takich lossów
1) Centre loss
2) Cosine loss
3) Arc loss



\paragraph{Implementacja}
Papier googla twierdzi ze udało im sie uzyskać dobry score jednak zadne inne prace nie
raportują sukcesywngeo otrzymania podobnego wyniku. Może to wynikać z 1) braku zbliżonego
wielkością zbioru danych użytych do uzyskania wyniku raportowangeo przez googla 2) brakiem
podobnych zasobów obliczeniowych

Została podjęta próba implementacji wytrenowania modelu stosując w.w.y metodę wykorzystując zbiór danych VggFace2. 

W celach pokazania, że implementacja jest poprawna porównujemy następujące modele

Porównanie:
dwa poejścia: dwa datasety, modele niekoniecznie moje ale testowane na własnej ewaluacji

 triplet(google report), wytrenowanych na danych prywatnych
 triplet(own), centre(external source) wytrenowanych na VggFace2
 triplet(own), centre(external source) wytrenowanych na MS1m
-> wyjdzie, że triplet loss jest o wiele mniej skuteczna 

see this:
https://arxiv.org/pdf/1901.08616.pdf
https://arxiv.org/pdf/1709.02940.pdf



\subsubsection{Rozporszona weryfikacja twarzy}
porównanie:

centre loss ms1m


globalny model arcface uczony na ms1m, testowany na lwf i megaface
globalny model arcface uczony na ms1m, dotrenowany tripletem na VggFace2, testowany


globalny model arcface dotrenowanie na lwf ale negatywne przykłady samplujemy z generatora
globalny model arcface dotrenowanie na megaface ale negatywne przykłady samplujemy z generatora



