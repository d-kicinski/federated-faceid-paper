%  Implementacja
% - datasets


\subsection{Implementacja systemu weryfikacji twarzy TODO}

\subsubsection{Zbiory danych}

W literaturze znajduje się duża liczba zbiorów do treningu ekstraktorów cech twarzy(cytaty).
Popularnym wyborem jest zbiór MS1M, posiada on duża różnorodność twarzy oraz duża liczbę
przykładów trenujących. VggFace2 jest relatywnie nowym zbiorem i nie tak popularnym jak poprzedni
ale cechuje się nie tylko dużym rozmiarem ale i dużą różnorodnością wewnątrz-klasową.

LFW jest jest niejako standardem w testowaniu systemów do weryfikacji, jednak ze względu na jego niewielkie rozmiary i niską różnorodność wewnątrz-klasową nie będzie odpowiedni do testowania systemu opartego o FL dlatego jego zastosowanie ograniczymy do walidacji pre-trenowanych  modeli trenowanych w tradycyjnym sposobem na serwerze.



\begin{table}[h]
\begin{center}
\begin{tabular}{c|c|c|c|c}
\hline
Zbiór danych  & \# osób   &   \# zdjęć  &   \# zdjęć na osobę   &   aligned \\
\hline
VGGFace2-train \cite{DatasetVGGFace2}     & 9.1K & 3.3M & 80/362.6/843 & Nie  \\ 
MS1M-DeepGlint \cite{DatasetGlintweb}   & 87K  & 3.9M & ?/44.8/? & Tak \\
\hline
%  Walidacja  & \# osób   &   \# zdjęć  &   \# zdjęć na osobę   &   aligned \\
\hline
LFW \cite{DatasetLFW}   & 5,749  & 13,233 & 1/2.3/530 & Nie \\
\hline
\end{tabular}
\end{center}
\caption{\textbf{Zbiory danych}.}
\label{table:dataset}
\vspace{-4mm}
\end{table}


Papier googla twierdzi ze udało im sie uzyskać dobry score jednak zadne inne prace nie
raportują skutecznego otrzymania podobnych rezultatów. Może to wynikać
\begin{itemize}
    \item braku zbliżonego wielkością zbioru danych użytych do uzyskania raportowanych wyników
    \item brakiem podobnych zasobów obliczeniowych
\end{itemize}

Została podjęta próba implementacji wytrenowania modelu stosując w.w.y metodę wykorzystując zbiór danych VggFace2. 

W celach pokazania, że implementacja jest poprawna porównujemy następujące modele

\subsubsection{Ewaluacja}


\subsubsection{Wyniki}

\begin{figure}[h]
  \centering
  \includegraphics[width=0.70\textwidth]{img/facenet_train.pdf}
  \caption{\fedavglong}%
  \label{fig:facenet_train}%
\end{figure}

%see this:
%https://arxiv.org/pdf/1901.08616.pdf
%https://arxiv.org/pdf/1709.02940.pdf


