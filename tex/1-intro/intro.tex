\newpage
\section{Wstęp}

Coraz częściej urządzenia internetu rzeczy stają się głównymi urządzeniami komputerowymi coraz
większej liczby użytkowników~\cite{SmartphoneOwenrship,SmartphoneOwenrshipv2}. Często gromadzą one wrażliwe dane i dostęp do takich
urządzeń przez niewłaściwe osoby grozi nieodwracalnymi stratami dla ich właściciela. Nowe
przyrządy wyposażone w odpowiednie sensory pozwalają na uwierzytelnienie dostępu już nie tylko za
pomocą hasła ale również przez weryfikacje biometryczną. Zabezpieczenia biometryczne mogą się
opierać na rozpoznawaniu linii papilarnych, głosu, skanowaniu żył, czy też tęczówki lub siatkówki
oka. W szczególności popularnym rozwiązaniem jest weryfikacja użytkownika przez biometrie twarzy ~\cite{FaceBiometic}.

Metody weryfikacji twarzy (ją jakieś) wymagają wyspecjalizowanych i dokładnych kamer (need fact
check). Z jednej strony montowanie drogich i nowoczesnych kamer na urządzeniach IoT do celów
weryfikacji jest nieopłacalne finansowo, a z drugiej z perspektywy użytkownika pożądane jest
posiadanie możliwie dokładnego systemu weryfikacji dostępu. Najnowocześniejsze i najbardziej
dokładne metody bazują w całości lub przynajmniej części na bazie sieci neuronowych (cite,fact
check). Metody te pewnym stopniu niwelują potrzebę posiadania wyspecjalizowanych kamer jednak
dokładność nowoczesnych metod jest bardzo uzależniona od jakości i ilość danych, które posłużyły
do wytrenowania sieci neuronowej.

Sensory, w które wyposażone są te urządzania (aparat, mikrofon, itp), w połączeniu z
faktem, że są używane codziennie, gromadzą niebywałą ilość, zazwyczaj prywatnych,
danych. Modele wyuczone na takich danych dają znakomitą poprawę ich użyteczności jednak ze względu
na ich wrażliwy charakter wiąże się to z ryzykiem i wysoką odpowiedzialnością ich
przechowywania w scentralizowanej lokalizacji albo nawet całkowitym brakiem dostępu do tych
danych.

Federated Learning pozwala na bezpieczne dla użytkownika wykorzystanie jego prywatnych danych
w celu dotrenowania sieci neuronowych i w tym poprawy ich jakości. W tej pracy zostanie
zbadana metoda ta metoda uczenia sieci neuronowych w implementacji systemu rozpoznawania twarzy
systemu na urządzania IoT.

\smallbreak

Praca została podzielona na rozdziały. W rozdziale~\ref{sec:general} został przedstawiony ogólny
zarys implementowanego systemu. Opisujemy czym charakteryzują się urządzania IoT się, na czym
polega zadanie weryfikacji użytkownika i pokazujemy, że zastosowanie podejścia Federated
Learningu do tego zadania jest odpowiednie. Następnie w rozdziale~\ref{sec:verification} w
szczegółach zostaje opisany problem weryfikacji użytkownika na podstawie cech biometrycznych jego
twarzy. Omawiane zostają dwa współczesne podejścia wykorzystujące głębokie sieci neuronowe, ich
wady i zalety względem wykorzystania w FL. Prezentujemy dokładniej wybrane przez nas podejście
wraz z naszą implementacją i otrzymanymi wynikami. Rozdział~\ref{sec:federated} został poświęcony
na omówienie Federated Learningu. Opisujemy cechy środowiska, w którym zastosowanie tej metody
treningu sieci daje największy zysk, prezentujemy projekt całego systemu zaczynając od protokołu
komunikacji, założeń co do urządzeń końcowych oraz serwera. Omawiamy algorytm Federated
Averaging oraz jego implementacje. Ostatecznie w rozdziale~\ref{sec:fedfaceid}prezentujemy zastosowanie FL do dotrenowania
ekstraktora cech w rozproszonym zbiorze danych. Na końcu w rozdziale~\ref{sec:summary} zostało
zawarte podsumowanie tej pracy wraz z propozycją jej dalszego rozwoju.