\newpage
\section{Wstęp}

\subsection{Motywacja}

Ostatnio kluczowym postępem w wielu dziedzinach wykorzystujących modele predykcyjne było
wykorzystanie metod opartych o głębokie sieci neuronowe. Skuteczność takich modeli zależy w dużej
mierze od jakości i ilości danych wykorzystanych ich wytrenowania. Niedawnym trendem jest umieszczanie modeli bezpośrednio na urządzaniach końcowych(i w tym urządzaniach IoT) \cite{EdgeFacebook}. Coraz częściej urządzenia internetu rzeczy stają się głównymi urządzeniami komputerowymi coraz większej liczby
użytkowników~\cite{SmartphoneOwenrship,SmartphoneOwenrshipv2} przez co gromadzą one ogromną ilość
danych. Niestety przez ich, zazwyczaj, prywatny charakter dostęp do nich przez niewłaściwe osoby grozi nieodwracalnymi stratami dla ich właściciela przez przez co użytkownicy tych urządzeń niechętnie zgadzają się na wgląd to tych danych przez firmy trzecie.

Moją motywacją do podjęcia tego tematu jest zainteresowanie tematyką Federated Learningu(FL). Ta
prosta w założeniach idea pozwala na bezpieczny dla użytkownika trening modeli neuronowych z
wykorzystaniem jego prywatnych danych. Dane nigdy nie wychodzą poza urządzenia użytkowników. W
tej pracy zostanie ona zastosowana do ostatnio popularnego rozwiązania weryfikacji użytkownika
urządzania końcowego przez biometrie twarzy~\cite{FaceBiometic}.

% Nowe przyrządy wyposażone w odpowiednie sensory pozwalają na
% uwierzytelnienie dostępu już nie tylko za pomocą hasła ale również przez weryfikacje
% biometryczną. Zabezpieczenia biometryczne mogą się opierać na rozpoznawaniu linii papilarnych,
% głosu, skanowaniu żył, czy też tęczówki lub siatkówki oka. W szczególności popularnym
% rozwiązaniem jest weryfikacja użytkownika przez biometrie twarzy

% Federated Learning pozwala na bezpieczne dla użytkownika wykorzystanie jego prywatnych danych
% w celu dotrenowania sieci neuronowych i w tym poprawy ich jakości. W tej pracy zostanie
% zbadana metoda ta metoda uczenia sieci neuronowych w implementacji systemu rozpoznawania twarzy
% systemu na urządzania IoT.

\subsection{Cel pracy}
Celem pracy było zaprojektowanie oraz zaimplementowanie systemu dotrenowywania neuronowych modeli ekstrakcji cech twarzy na urządzeniach IoT. Powinien on pozwalać na ulepszenie istniejących modeli wytrenowanych w tradycyjny sposób poprzez dodatkowy trening na urządzeniach końcowych użytkowników z wykorzystaniem przechowywanych na nich prywatnych danych użytkownika. Projektowany system nie powinień ujawniać danych użytkowników oraz ma być kompatybilny z istniejącymi algorytmami zabezpieczającymi~\cite{FLSecureAggregation}.

\subsection{Zakres pracy}
Niniejsza praca obejmuje
\begin{itemize}
    \item  projekt i implementacje uproszczonej wersji systemu FL współpracującej z systemami biometrycznej weryfikacji użytkownika,
    \item przegląd, implementacje i trening modeli neuronowych do ekstrakcji cech twarzy,
    \item walidacje systemu oraz dobór jego parametrów.
\end{itemize}

\subsection{Struktura pracy}
Praca została podzielona na siedem rozdziałów. W rozdziale~\ref{sec:general} został przedstawiony ogólny zarys implementowanego systemu. Opisujemy czym charakteryzują się urządzania IoT, na czym
polega zadanie weryfikacji użytkownika i pokazujemy, że zastosowanie podejścia Federated
Learningu do tego zadania jest odpowiednie. Następnie w rozdziale~\ref{sec:verification} w
szczegółach zostaje opisany problem weryfikacji użytkownika na podstawie cech biometrycznych jego
twarzy. Omawiane zostają dwa współczesne podejścia wykorzystujące głębokie sieci neuronowe, ich
wady i zalety względem wykorzystania w FL. Prezentujemy dokładniej wybrane przez nas podejście
wraz z naszą implementacją i otrzymanymi wynikami. Rozdział~\ref{sec:federated} został poświęcony
na omówienie Federated Learningu. Opisujemy cechy środowiska, w którym zastosowanie tej metody
treningu sieci daje największy zysk, prezentujemy projekt całego systemu zaczynając od protokołu
komunikacji, założeń co do urządzeń końcowych oraz serwera. Omawiamy algorytm Federated
Averaging oraz jego implementacje. Ostatecznie w rozdziale~\ref{sec:fedfaceid} prezentujemy zastosowanie FL do dotrenowania
ekstraktora cech w rozproszonym zbiorze danych. Na końcu w rozdziale~\ref{sec:summary} zostało
zawarte podsumowanie tej pracy wraz z propozycją jej dalszego rozwoju.