\newpage
\section[Wstęp]{Wstęp}

Coraz częściej urządzenia internetu rzeczy stają się głównymi urządzeniami komputerowymi coraz
większej liczby użytkowników (cite). Często gromadzą one wrażliwe dane i dostęp do takich
urządzeń przez niewłaściwe osoby grozi nieodwracalnymi stratami dla ich właściciela. Nowe
przyrządy wyposażone w odpowiednie sensory pozwalają na uwierzytelnienie dostępu już nie tylko za
pomocą hasła ale również przez weryfikacje biometryczną. Zabezpieczenia biometryczne mogą się
opierać na rozpoznawaniu linii papilarnych, głosu, skanowaniu żył, czy też tęczówki lub siatkówki
oka. W szczególności popularnym rozwiązaniem jest weryfikacja użytkownika przez biometrie twarzy
(jakiś cytat).

Metody weryfikacji twarzy (ją jakieś) wymagają wyspecjalizowanych i dokładnych kamer (need fact
check). Z jednej strony montowanie drogich i nowoczesnych kamer na urządzeniach IoT do celów
weryfikacji jest nieopłacalne finansowo, a z drugiej z perspektywy użytkownika pożądane jest
posiadanie możliwie dokładnego systemu weryfikacji dostępu. Najnowocześniejsze i najbardziej
dokładne metody bazują w całości lub przynajmniej części na bazie sieci neuronowych (cite,fact
check). Metody te pewnym stopniu niwelują potrzebę posiadania wyspecjalizowanych kamer jednak
dokładność nowoczesnych metod jest bardzo uzależniona od jakości i ilość danych, które posłużyły
do wytrenowania sieci neuronowej.

Sensory, w które wyposażone są te urządzania (na przykład aparat, mikrofon, itp), w połączeniu z
faktem, że są używane codziennie, gromadzą niebywałą ilość, zazwyczaj prywatnych,
danych. Modele wyuczone na takich danych dają znakomitą poprawę ich użyteczności jednak ze względu
na wrażliwy charakter danych wiąże się z ryzykiem i wysoką odpowiedzialnością ich
przechowywania w scentralizowanej lokalizacji albo nawet całkowitym brakiem dostępu do tych
danych.

Federated Learning pozwala na bezpieczne dla użytkownika wykorzystanie jego prywatnych danych
(zdjęć) w celu dotrenowania sieci neuronowych i w tym poprawy ich jakości. W tej pracy zostanie
zbadana metoda ta metoda uczenia sieci neuronowych w implementacji systemu rozpoznawania twarzy
systemu na urządzania IoT.

Główne kontrybucje tej pracy to 1) Implementacja i weryfikacja algorytmu \fedavglong dla zadań klasyfikacji obrazów oraz weryfikacji twarzy.
