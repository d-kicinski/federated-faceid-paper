\newpage
\section[Wstęp]{Wstęp}

Coraz częściej urządzenia internetu rzeczy stają się głównymi urządzeniemi komputerowymi(cytat).
Sensory, w które wyposarzone są te urządznia (takie jak aparat, mikrofon, GPU), w połączniu z
faktem, że urządznia te są używane codzienie, gromadzą niebywałą ilość, zazwyczaj prywatnych,
danych. Modele wyuczone na takich danych dadzą znakomitą poprawę użyteczności jednak ze względu
na wrażliwy charakter danych wiąze się z ryzykiem i wysoką odpowiedzialnością ich
przechechowywania w scentralizowanej lokalizacji albo nawet całkowitym brakiem dostępu do tych
danych.




Urządznia IoT często gromadzą wrażliwe dane i dostęp do takich urządznien przez niewłaściwe osoby
grozi nieodwracalnymi stratami dla właściciela urządznia. Nowe urządzania wyposarzone w
odpowiednie sensory pozwalają na uwirzytelnienie dostępu nie tylko po haśle ale i przez
weryfikacje biometryczną. Zabezpieczenia biometryczne mogą się opierać również na rozpoznawaniu
linii papilarnych, głosu, skanowaniu żył, czy też tęczówki lub siatkówki oka. W szeczólności
popularnym rozwiązaniem jest weryfikcja urzytkownika przez biometrie twarzy (jakis cytat).

W tej pracy zostanie zbadana metoda uczenia opisana w (cite) w implementacji systemu
rozpoznawania twarzy systemu na urządznia IoT.


Główne kontrybucje tej pracy to 1) Implementacja i weryfikacja algorytmu \fedavglong dla zadań klasyfikacji obrazów oraz weryfikacji twarzy
