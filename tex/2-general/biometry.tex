\subsection{Uwierzytelnienie użytkownika biometrią twarzy}

Tymczasem wiele rodzajów uwierzytelniania jest wadliwych, zwłaszcza hasła. Ponieważ przeciętny człowiek musi śledzić wiele haseł, z których wiele może wymagać regularnej aktualizacji, hasło stało się niepraktyczne i niepewne.

Przestępcy mogą „wyłudzić” ludzi z ich haseł. Lub mogą użyć oprogramowania „brute force”, aby
wypróbować miliony kombinacji słów co sekundę. Problem polega na tym, że obecnie nie istnieje
szeroko rozpowszechniona, uniwersalna (i wygodna) forma tożsamości cyfrowej. Odpowiedzią na to
może być uwierzytelnienie biometrią twarzy:

\begin{itemize}
    \item Jest nieinwazyjny, ponieważ użytkownik nie wymaga fizycznej interakcji
    \item Jest stosunkowo łatwy do wdrożenia i wdrożenia
    \item Koszty technologii - kamery, przetwarzanie - spadają
    \item Masowa adopcja przez producentów smartfonów sprawiła, że stało się to znane użytkownikom
    \item Jego wyniki są dokładne i szybkie
\end{itemize}

Z tego powodu twórcy smartfonów i tabletów używają teraz identyfikacji twarzy jako domyślnej metody „odblokowania” dla swoich urządzeń i usług. Rzeczywiście, Counterpoint Research spodziewa się, że w 2020 r. Pojawi się ponad miliard smartfonów z rozpoznawaniem twarzy.

Dzięki ulepszeniom technologii aparatu, procesom mapowania, uczeniu maszynowemu i szybkościom przetwarzania, rozpoznawanie twarzy z biegiem lat.

Większość systemów wykorzystuje technologię kamery 2D, która tworzy płaski obraz twarzy i mapuje „punkty węzłowe” (rozmiar / kształt oczu, nosa, kości policzkowych itp.). Następnie system oblicza względną pozycję węzłów i przekształca dane w kod numeryczny. Algorytmy rozpoznawania przeszukują zapisaną bazę danych twarzy w celu znalezienia dopasowania.

Technologia 2D działa dobrze w stabilnych, dobrze oświetlonych warunkach, takich jak kontrola paszportowa. Ale jest mniej skuteczny w ciemniejszych przestrzeniach i nie może zapewnić dobrych rezultatów, gdy obiekty się poruszają. Zdjęcie jest łatwe do sfałszowania.

Dzisiaj wcześniej zaawansowane procesy trafiają do urządzeń na rynku masowym. Na przykład, Apple wykorzystuje technologię kamery 3D do zasilania funkcji Face ID opartej na podczerwieni w swoim iPhonie X. Zdjęcia termiczne IR odwzorowują wzory twarzy pochodzące głównie ze wzoru powierzchownych naczyń krwionośnych pod skórą.
 
Kolejnym kluczowym postępem jest „głęboka splotowa sieć neuronowa”. Jest to rodzaj uczenia maszynowego, w którym model znajduje wzorce w danych obrazu. Wykorzystuje sieć sztucznych neuronów, które imitują funkcjonowanie ludzkiego mózgu. W efekcie sieć zachowuje się jak czarna skrzynka. Podano wartości wejściowe, których wyniki nie są jeszcze znane. Następnie sprawdza, czy sieć przynosi oczekiwany wynik. Gdy tak nie jest, system dokonuje regulacji, dopóki nie zostanie poprawnie skonfigurowany i będzie w stanie systematycznie generować oczekiwane wyniki.