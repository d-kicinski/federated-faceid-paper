\newpage
\section[general]{Modele neuronowe, a urządzenia końcowe}\label{sec:general}

Tworzenie modeli predykcyjnych typu sieć neuronowa opiera się na odpowiednim ustaleniu wag
wybranej architektury sieci. Taki model jest parametryzowany wagami - liczbami
zmiennoprzecinkowymi, które na podstawie przykładów trenujących są iteracyjnie poprawione przez
wybrany algorytm optymalizacji. Proces ten nazywany w literaturze jest treningiem. Po wytrenowaniu model trafia do środowiska produkcyjnego, w którym będzie dawał predykcje.

Przekazanie modelu do wykorzystania dla użytkowników można zrealizować na kilka sposobów. Sposób w jaki jest to wykonywanie zależy w dużej mierze od architektury środowiska produkcyjnego oraz typu urządzeń z jakich będą korzystali użytkownicy. W ramach wstępu do tematyki tej pracy uwaga zostanie zwrócona na właśnie te dwa zagadnienia.

% Federated Learning jest naturalną kontynuacją ostatnich trendów w dziedzinie treningu i deploymentu modeli neuronowych do środowisk produkcyjnych dlatego na wstępnie zostaną przedstawione typo