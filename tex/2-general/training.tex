\subsection{Wdrażanie modeli neuronowych na urządzenia końcowe}

Popularnym i prostym w implementacji środowiskiem produkcyjnym jest wdrożenie modelu w postaci
serwisu chmurowego~\cite{ServerFacebook}. Serwis postawiony na serwerze wystawia dla konsumentów
swoje sieciowe API. Na zapytanie o predykcje dla danych wejściowych serwer odsyła wyjście
zwrócone przez model - rysunek~\ref{fig:deploy-0}. Rozwiązanie to oczywistą zaletę - inferencja
modelu wykonywana jest w chmurze przez co urządzenie użytkownika nie zużywa swoich zasobów obliczeniowych oszczędzając energię oraz pamięć, jednak kosztem wymagania stałego dostępu urządzenia do internetu oraz potrzebą wysłania danych wejściowych do serwera.

\begin{figure}[h!]
    \centering
    \includegraphics[width=0.8\linewidth]{img/ml_server_0_drawio.pdf}
    \caption{Środowisko produkcyjne z serwisem chmurowym.}
    \label{fig:deploy-0}
    \vspace{-4mm}
\end{figure}

Innym, coraz to bardziej popularnym, podejściem jest wdrożenie modeli bezpośrednio na urządzenia
końcowe~\cite{EdgeFacebook}(rysunek~\ref{fig:deploy-1}). Jest to proces o tyle bardziej
skomplikowany przez to, że aktualnie popularne frameworki\cite{PyTorch,Tensorflow,Mxnet} do
budowania i trenowania sieci neuronowych nie wspierają większości platform, na których budowane
są urządzenia IoT. Taki sposób wdrożenia modelu pozwala na korzystanie z modelu w trybie offline,
zwalnia nas z obowiązku utrzymywania serwera produkcyjnego oraz nie zmusza użytkownika do
wysyłania danych na zewnątrz urządzenia. Niestety narzucamy przez to duże ograniczenia na model
oraz na docelowe urządzenie IoT. Urządzenia końcowe mają o wiele niższe możliwości obliczeniowe
oraz pamięciowe dlatego wdrażany model powinien być dostatecznie mały tak aby zbytnio nie
obciążać urządzenia. Aktualnie urządzenia zdolne do uruchomienia sieci wyprodukowanych przez
popularniejsze frameworki to wszystkie platformy obsługujące system linux oraz smarfony z
systemem Android lub iOS.

\begin{figure}[h!]
    \centering
    \includegraphics[width=0.5\linewidth]{img/ml_server_1_drawio.pdf}
    \caption{Środowisko produkcyjnego - wdrożenie modelu bezpośrednio na urządzenia końcowe.}
    \label{fig:deploy-1}
    \vspace{-4mm}
\end{figure}

Na rysunku~\ref{fig:deploy-2} przedstawiono schematyczne dotrenowanie modeli na urządzeniach
końcowych na danych znajdujących sie na urządzeniach. Wykorzystujemy w ten sposób zbierane przez
urządzenia dane i aktualizujemy wagi modelu wykorzystując dane pochodzące bezpośrednio od
użytkownika. Urządzenie zyskuje w ten sposób model
lepiej spersonalizowany pod swoją dystrybucje danych wejściowych. Modele nie są współdzielone,
każdy z użytkowników ma swóją wersje modelu.

\begin{figure}[h!]
    \centering
    \includegraphics[width=0.8\linewidth]{img/ml_server_2_drawio.pdf}
    \caption{Wdrożenie na urządzenia końcowe i dotrenowanie modelu wykorzystując lokalnie zbierane dane}
    \label{fig:deploy-2}
    \vspace{-4mm}
\end{figure}

Na rysunku~\ref{fig:deploy-3} przedstawione zostało podejście Federated Learningu. Od poprzednika
różni go tylko element serwera FL, który pobiera dotrenowane modele od urządzeń agreguje je i
rozsyła je z powrotem. Podejście to pozwala na lepszą generalizacje modeli i w efekcie lepszą
predykcję. Niestety przez wprowadzenie serwera pośredniczącego w wymianie modeli cały system
ulega znacznemu skomplikowaniu i wymagany jest odpowiedni protokół dotrenownywania, agregacji i
wymiany modeli. W szczegółach to podejście zostanie omówione w rozdziale~\ref{sec:federated}.

\begin{figure}[h!]
    \centering
    \includegraphics[width=1\linewidth]{img/ml_server_3_drawio.pdf}
    \caption{Wdrożenie modelu i dalsze jego dotrenowywanie stosując podejście Federated Learning}
    \label{fig:deploy-3}
    \vspace{-4mm}
\end{figure}
