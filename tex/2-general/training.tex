\subsection{Wdrażanie modeli neuronowych na urządzenia końcowe}

Tworzenie modeli predykcyjnych typu sieć neuronowa opiera się na odpowiednim ustaleniu wag
wybranej architektury sieci. Taki model jest parametryzowany wagami - liczbami
zmiennoprzecinkowymi, które na podstawie przykładów trenujących są iteracyjnie poprawione przez
wybrany algorytm optymalizacji. Proces ten nazywany w literaturze jest treningiem. Po wytrenowaniu model trafia do środowiska produkcyjnego, w którym będzie dawał predykcje.

Popularnym i prostym w implementacji środowiskiem produkcyjnym jest wdrożenie modelu w postaci
serwisu chmurowego. Serwis postawiony na serwerze wystawia API. Na zapytanie o predykcje dla
danych wejściowych serwer odsyła predykcje zwróconą przez model - rysunek~\ref{fig:deploy-0}.
Rozwiązanie to oczywistą zaletę inferencja modelu wykonywana jest w chmurze przez co urządzenie
nie zużywa swoich zosowów obliczeniowych oszczędzająć energię oraz pamięć, jednak kosztem
wymagania stałwgo dostępu urządzenia do internetu oraz potrzebą wysłania dancyh wejściowych do
serwera.

\begin{figure}[h!]\label{fig:deploy-0}
    \centering
    \includegraphics[width=0.8\linewidth]{img/ml_server_0_drawio.pdf}
    \caption{Środowisko produkcyjne z serwisem chmurowym.}
    \vspace{-4mm}
\end{figure}

Innym, coraz to bardziej popularnym, podejściem jest wdrożenie modeli bezpośrenio na urządzenia
końcowe~\cite{EdgeFacebook}(rysunek~\ref{fig:deploy-1}). Jest to proces o tyle bardziej skomplikowany przez to, że aktualnie
popularne frameworki\cite{PyTorch,Tensorflow} do budowania i trenowania sieci neuronowych nie
wspierają większośći platform, na których budowane są urządzenia IoT. Taki sposób wdrożenia
modelu pozwala na korzystanie z modelu w trybie offline, zwalnia nas z obowiązku utrzymywania
serwera produkcyjnego oraz nie zmusza użytkownika do przesyłania danych na zewnątrz urządzenia.
Niestety narzucamy przez to duże ograniczenia na model oraz na docelowe urządzenie IoT.
Urządzenia końcowe mają o wiele niższe możliwości obliczeniowe oraz pamięciowe dlatego wdrażany
model powinien być dostatecznie mały tak aby zbytnio nie obciążać urządzenia. Aktualnie
urządzenia zdolne do uruchomienia sieci wyprodukowanych przez popularniejsze frameworki to
wszystkie platformy obsługujące system linux oraz smarfony z systeme Android oraz iOS.

\begin{figure}[h!]\label{fig:deploy-1}
    \centering
    \includegraphics[width=0.5\linewidth]{img/ml_server_1_drawio.pdf}
    \caption{Środowisko produkcyjnego - wdrożenie modelu bezpośrednio na urządzenia końcowe.}
    \vspace{-4mm}
\end{figure}

Na rysunku~\ref{fig:deploy-2} przedstawio schematyczne dotrenowywanie modeli na urządzeniach
koncowych na danych znajdujących sie na urządzeniach. Utilizujemy w ten sposób zbierane przez urządznia dane i aktualizujemy wagi modelu wykorzystując dane pochodzące bezpośrednio od użytkownika. Urządznie zyskuje w ten sposób model
lepiej spersonalizowany pod swoją dystrybucje danych wejściowych. Modele nie są współdzielone,
każdy z użytkowników ma swóją wersje modelu.

\begin{figure}[h!]\label{fig:deploy-2}
    \centering
    \includegraphics[width=0.8\linewidth]{img/ml_server_2_drawio.pdf}
    \caption{Wdrożenie na urządzenia końcowe i dotrenowanie modelu wykorzystując lokalnie zbierane dane}
    \vspace{-4mm}
\end{figure}

Na rysunku~\ref{fig:deploy-3} przedstawione zostało podejście Federated Learningu. Od poprzednika
różni go tylko element serwera FL, który pobiera dotrenowane modele od urządzeń agreguje je i
rozsyła je spowrotem. Podejście to pozwala na lepszą generalizacje modeli i w efekcie lepszą
predykcję. Niestety przez wprowadzenie serwera pośredniczącego w wymianie modeli cały system
ulega znaczem skomplikowaniu i wymagany jest odpowiedni protokół dotrenownywania, agregacji i
wymiany modeli. W szczegółach to podejście zostanie omówione w rozdziale~\ref{sec:federated}.

\begin{figure}[h!]\label{fig:deploy-3}
    \centering
    \includegraphics[width=1\linewidth]{img/ml_server_3_drawio.pdf}
    \caption{Wdrożenie modelu i dalsze jego dotrenowywanie stosując podejście Federated Learning}
    \vspace{-4mm}
\end{figure}
